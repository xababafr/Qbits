%
% PENSER À :
% - vérifier taille de police et interligne
%

\documentclass[a4paper,12pt]{report}

\usepackage[utf8]{inputenc}
\usepackage[T1]{fontenc}
\usepackage[francais]{babel}

\usepackage[a4paper,top=3cm,bottom=3cm]{geometry}

\usepackage{graphicx}
\usepackage{pdfpages}

\usepackage{fancyhdr}
\pagestyle{fancy}
\renewcommand\headrulewidth{1pt}
\rhead{ \LaTeX }
\cfoot{ \thepage }

\usepackage{titlesec}

% maths
\usepackage{amsmath}
\usepackage{amssymb}
\newcommand{\prodSc}[2]{\langle #1 / #2 \rangle}

\titleformat{\chapter}{\normalfont\huge}{\thechapter.}{20pt}{\huge}

\renewcommand{\chaptername}{ }
\DeclareTextFontCommand{\emph}{\bfseries}

\newcommand{\para}[1]{\par{#1}\\}

\title{QUBITS}
\author{QUÉLARD Xavier}
\date{ \today{} }

\makeindex

\begin{document}

%
% -----------------------
% [1] PAGE DE GARDE
% -----------------------
%

%\includepdf{Garde}
\maketitle

\chapter*{Résumé}

\para{
	...
}

\tableofcontents

\chapter*{Introduction}
\addcontentsline{toc}{chapter}{Introduction}

\para{
	...
}

\chapter{Historique : mécanique quantique et QuBit}

\para{
	...
}

\chapter{Comment réaliser un QuBit?}
	\section{Photons polarisés}

\para{
	...
}

	\section{Pièges à ions / technologie NMR}

\para{
	...
}

	\section{Pièges à ions / technologie NMR}

\para{
...
}

	\section{Solid State QuBit (supraconductor technology)}

\para{
...
}

\chapter{Simulation informatique}
	\section{Rappels mathématiques}
		\subsection{Loi de composition interne et groupe}

\par{
Une loi de composition interne $\star{}$ sur l'ensemble $X$ est une aplication de la forme : \[\star : X \times X \rightarrow X\]
}

\par{
	Soit $G$ un ensemble non vide, muni d'une loi de composition interne \oplus. $ (G, \oplus)$ est un groupe abélien \Leftrightarrow
}

\begin{itemize}
\item[$\bullet$] $\oplus$ associative : $\forall (x,y) \in G^2, (x \oplus y) \oplus y = x \oplus ( y \oplus z )$
\item[$\bullet$] $\exists e \in G / x \oplus e = e \oplus x = x$ (e est le neutre du groupe G selon la loi $\oplus$)
\item[$\bullet$] $\oplus$ commutative : $\forall (x,y) \in G^2, x \oplus y = y \oplus x$ \\
\end{itemize}

		\subsection{Anneaux et corps}

\par{
	Disposant de la définition d'un corps commutatif, nous pouvons maintenant donner la définition d'un anneau commutatif. Soit $A$ un ensemble non vide, muni de deux lois de compositions interne \oplus et \otimes.
}

\par{
	Alors $(A, \oplus, \otimes)$ anneau commutatif $\Leftrightarrow$
}

\begin{itemize}
\item[$\bullet$] $ (A, \oplus)$ est un groupe commutatif
\item[$\bullet$] la loi $\otimes$ est associative
\item[$\bullet$] la loi $\otimes$ est distributive par rapport à la loi $\oplus$, c'est-à-dire que : \[ \forall (x,y,z) \in A^3, (x \oplus y) \otimes z = x \otimes ( y \oplus z ) \]
\item[$\bullet$] la loi $\otimes$ est commutative : $\forall (x1,x2) \in A^2, x1 \otimes x2 = x2 \otimes x1$
\end{itemize}

\vspace{1\baselineskip}

\par{
	Un corps commutatif est alors simplement un anneau commutatif dont tous les éléments sont inversibles, exceptés le neutre pour l'opération $\oplus$. Il est alors aisé de remarquer que l'ensemble $\mathbb{R}$ ou bien $\mathbb{C}$ sont, munis des opérations $(+,\times)$, des corps.
}

		\subsection{espace vectoriel et produit scalaire}

\par{
	Soit $E$ un ensemble non vide et $(\mathbb{K},+,\times)$ un corps de neutre $0_{\mathbb{K}}$ pour $+$ et $1_{\mathbb{K}}$ pour $\times$. On note $(E,+,\cdot)$ l'ensemble $E$ muni de la même loi $+$ que $\mathbb{K}$ (c'est donc une loi interne à $E$), et d'une loi externe $\cdot : \mathbb{K} \times E \rightarrow E$ .
}

\par{
	Alors $E$ est un espace vectoriel $\Leftrightarrow$
}

\begin{itemize}
\item[$\bullet$] $ (E, +)$ est un groupe commutatif
\item[$\bullet$] $\forall \lambda \in \mathbb{K}, \forall (x,y) \in E^2, \lambda \cdot (x+y) = (\lambda \cdot x) + (\lambda \cdot y)$ (distributivité)
\item[$\bullet$] $\forall (\lambda, \mu) \in \mathbb{K}^2, \forall x \in E, (\lambda + \mu) \cdot x = \lambda \cdot x + \mu \cdot x$
\item[$\bullet$] $\forall (\lambda, \mu) \in \mathbb{K}^2, \forall x \in E, (\lambda \times \mu) \cdot x = \lambda \cdot (\mu \cdot x)$
\item[$\bullet$] $\forall x \in E, 1_{\mathbb{K}} \cdot x = x$
\end{itemize}

\vspace{1\baselineskip}

\par{
	On appele les éléments de $E$ des vecteurs, les éléments de $\mathbb{K}$ des scalaires, et le vecteur $0_{\mathbb{K}}$ est appelé le vecteur nul. En résumé, un espace vectoriel est un espace E consitué d'éléments appelés vecteurs, qui sont stables par addition et par multiplication d'un scalaire. Les espaces $(\mathbb{R},+,\cdot)$ et $(\mathbb{C},+,\cdot)$ sont donc des espaces vectoriels (le second est appelé espace vectoriel complexe).
}

\vspace{1\baselineskip}

\par{
	On appele produit scalaire sur $E$ (espace vectoriel) toute forme bilinéaire symétrique et définie positive, c'est-à-dire :
}

\begin{itemize}
\item[$\bullet$] forme : c'est une application du type
$\prodSc{\cdot}{\cdot} : \left\{
  \begin{array}{rcr}
    E \times E \rightarrow \mathbb{K} \hfill \\
    (u,v) \mapsto \langle u / v \rangle \\
  \end{array}
\right$
\item[$\bullet$] symétrie : $ \forall (u,v) \in E^2, \prodSc{u}{v} = \prodSc{v}{u} $
\item[$\bullet$] linéarité (de la symétrie découle alors la bi-linéarité) : $\forall (u,v,w) \in E^3, \proSc{u}{v+w}= \prodSc{u}{v} + \prodSc{u}{w}$
\item[$\bullet$] defini positif : $\forall u \in E, \proSc{u}{u} \geq 0$ et $\forall u \in E, \prodSc{u}{u} = 0 \Lefrightarrow u = 0_{E}$
\end{itemize}

\vspace{1\baselineskip}

\par{
	Il est important de remarquer que s'il l'on se place dans un espace vectoriel complexe, le produit scalaire donne un nombre complexe, tandis qu'en se plaçant dans un espace vectoriel sur le corps des réels, le produit scalaire donnera lui même un réel. De manière générale, il associe à vecteur un élément du corps $\mathbb{K}$.
}

		\subsection{norme induite et espace de Hilbert}

\par{
	Une norme est une application $N : E \rightarrow \mathbb{R}_{+}$ et qui satisfait l'hypothèse de séparation ($\forall u \in E, N(u) = 0 \Rightarrow u = 0_{E}$), d'absolue homogénéité ($\forall \lambda \in \mathbb{K}, \forall u \in E, N(\lambda \cdot u) = \lambda \cdot N(u)$), et l'inégalité triangulaire ($\forall (u,v) \in E^2, N(u+v) \leq N(u) + n(v)$).
}

\vspace{1\baselineskip}

\par{
	A chaque produit scalaire est associé une norme, que l'on appele norme induite par le produit scalaire. Elle est définie par :
}

$$
N( \cdot ) : \left\{
  \begin{array}{rcr}
    E \rightarrow \mathbb{R}_{+} \hfill \\
    u \mapsto \sqrt{\prodSc{u}{u}} \\
  \end{array}
\right
$$

	\section{Application aux QuBits ( + sphere de bloch)}

	\section{Portes logiques}

%
% -----------------------
% [?] CONCLUSION
% -----------------------
%/

\chapter*{Conclusion}
\addcontentsline{toc}{chapter}{Conclusion}

\para{
	...
}

\chapter*{Bibliographie}

\para{
	...
}


\chapter*{Annexe}

\end{document}
